\documentclass[11pt]{article}

\usepackage[round]{natbib}
% for tikz pic
\usepackage{tikz}
\usetikzlibrary{
  decorations.pathreplacing,
  calc,
  patterns,
  math,
  external,
  spy}


\usepackage[utf8]{inputenc}
\usepackage[a4paper]{geometry}
\usepackage{standalone}


\usepackage{graphicx}
\usepackage{amsmath,amssymb,mleftright}
\usepackage{mathtools} % For cases environment
\usepackage{microtype}


\usepackage[british]{babel}

% Make a title for your question and provide your name (or a pseudonymn)
\title{Black-Scholes simplified}
\author{}
\date{}

\begin{document}
\maketitle

\section{Question}

Consider an at the money call option, expiring in one month.
It has a strike price of $K$, and the current price of the underling is $S$, where $S \approx K$.
The value of this option, using Black-Scholes and assuming that the risk-free rate of return is small and no dividents are due, is
\begin{align*}
S N(d_1)
- K e^{-r(T-t)} N(d_2)
\end{align*}
where
\begin{align*}
 d_1 =
 \frac{
      \log\left( \frac{S}{K} \right)
      + \left( r + \frac{1}{2}\sigma^2  \right)(T - t)
 }{
 \sigma \sqrt{T - t}
 }
\end{align*}
and
\begin{align*}
 d_2 = d_1 - \sigma \sqrt{T - t}
 \text{.}
\end{align*}
Given the special case where $S \approx K$ and one month to expiry, $(T-t) = 1/12$, how can we simplify this equation into something that can be quickly done on the back of an envelope by traders?


\section{Answer}

This question is asking you to prove that the price of a call option can be approximated by
\begin{align*}
  S \sigma \sqrt{\frac{T - t}{2\pi} }
\text{.}
\end{align*}
We don't require rigour, a heuristic proof will do.
Start with the price of a call option
\begin{align}
\label{eq:calloption:price}
S N(d_1)
- K e^{-r(T-t)} N(d_2)
\end{align}
and note that $e^{-r(T-t)} \approx 1$ when $r \approx 0$ and $(T-t)$, the time-to-expiry, are small.
Also substitute $K$ with $S$, leaving
\begin{align} \label{eq:approxfun:1}
S \times ( N(d_1) -  N(d_2) )
\text{.}
\end{align}
Next, consider $d_1$, and again use $K \approx S$ and $r \approx 0$, then
\begin{align*}
 d_1
 &=
 \frac{
      \log\left( \frac{S}{S} \right)
      + \left( 0 + \frac{1}{2}\sigma^2  \right)(T - t)
 }{
 \sigma \sqrt{T - t}
 } \\
 &=
 \frac{
      \left(\frac{1}{2}\sigma^2  \right)(T - t)
 }{
 \sigma \sqrt{T - t}
 } \\
 &=
 \frac{1}{2} \sigma \sqrt{T - t}
\\
 &= d_1^*
\text{.}
\end{align*}
Then you have
\begin{align*}
 d_2^* &= d_1^* - \sigma \sqrt{T - t} \\
       &=  -  \frac{1}{2} \sigma \sqrt{T - t} \\
       &=  - d_1^*
\end{align*}
which reduces
\eqref{eq:approxfun:1} to
\begin{align} \label{eq:approxfun:2}
S \times ( N(d_1^*) -  N(-d_1^*) )
\text{.}
\end{align}
Consider the function $N(x)$; the cumulative density function of the standard normal distribution.
We can approximate it using the Taylor series.
Let $N'(x) = f(x)$ be the probability density function of the standard normal distribution, then
\begin{align*}
N(x - a) = N(a)
          + f(a)(x - a)
          + \frac{1}{2!}f'(a)(x - a)^2
          + \frac{1}{3!}f''(a)(x - a)^3
\text{.}
\end{align*}
Only consider the first order approximation:
\begin{align*}
N(x - a) = N(a)
          + f(a)(x - a)
\end{align*}
and since
\begin{align*}
f(x) =
     \frac{1}{ \sqrt{ 2\pi } }
     e^{ - \frac{1}{2} (x)^2 }
\end{align*}
you have
\begin{align*}
N(d_1^*)&= N(d_1^* - 0) \\
            &= N(0) + f(0)(d_1^* - 0) \\
            &= \frac{1}{2} +  \frac{1}{ \sqrt{ 2\pi } }
            (d_1^* )
\end{align*}
and
\begin{align*}
N(-d_1^*) &= \frac{1}{2} +  \frac{1}{ \sqrt{ 2\pi } } (- d_1^* )
\text{.}
\end{align*}
Putting this into \eqref{eq:approxfun:2} yields
\begin{align*}
&\phantom{{}={}}
S \times \left(
\frac{1}{2} +  \frac{1}{ \sqrt{ 2\pi } } ( d_1^* ) -
\left(
\frac{1}{2} +  \frac{1}{ \sqrt{ 2\pi } } (- d_1^* )
\right)
\right)
\\
&=
S \times \left(
    \frac{2}{ \sqrt{ 2\pi } } ( d_1^* )
\right)
\\
&=
S \times \left(
    \frac{ 2 }{ \sqrt{ 2 \pi } }
    \left(
    \frac{1}{2} \sigma \sqrt{T - t}
    \right)
\right)
\\
&=
S \times \left(
     \sigma
    \sqrt{\frac{ T - t }{  2 \pi  }}
\right)
\text{,}
\end{align*}
which is what we can let the traders use to quickly approximate the price of a call option.
The accuracy of this approximation is shown in the figure below, drawn using
the values
$K = 80$,
$r = 0$,
$T-t = 30/365$, and
$\sigma=0.2$.
%
\begin{figure}[!htb]
\begin{center}
\documentclass{standalone}

%https://truonglatex.wordpress.com/2011/11/28/create-standalone-figures-which-are-input-able/
% if you want to input{} this into the maindocument, make sure you do \usepackage{standalone} in there
% That would ignore everything in the preamble of this file
% Makes it easier to maintain, since you can come back and recompile this figure

\usepackage{tikz}


\usetikzlibrary{
  decorations.pathreplacing,
  calc,
  patterns,
  math,
%  datavisualization.formats.functions,
  external,
  spy}

\begin{document}

% >=latex is the specificatin of the arrow tips
% See section 16.4.4 Defining Shorthands
% See pgfmanula 89.3 Syntax for Mathematical Expressions: Functions
% for a set of mathematicsal functions you can use

\def\K{80}
\def\r{0.00}
\def\ttm{30/365}
\def\ssigma{0.2}
\begin{tikzpicture}[
declare function={
          upside(\x)= max(0, \x);
          },
declare function={
          d1(\x)= (ln(\x/\K) + (\r + 0.5*(\ssigma*\ssigma))*(\ttm) )/( \ssigma*sqrt(\ttm));
          },
declare function={
          d2(\x)= d1(\x) - \ssigma * sqrt(\ttm);
          },
declare function={
          % This function guards against the error "margins too large"
          % It ensures x is in [a, b]
          bounded_x(\x,\a,\b) = min(max(\x,\a),\b);
          },
declare function={
          % This function ensures we never send anything with more than 3 decimal
          % places to exp
          eexp(\x)= exp(bounded_x(\x, -6, 6));
          },
declare function={
          % An approximation that will suffice for plotting purposes
          %Ncdf(\x)= 1/(1+eexp(-0.07056*(\x * \x * \x) - 1.5976* (\x) );
          Ncdf(\x)= 1/(1+eexp(-0.07056*(\x*\x*\x) - 1.5976* (\x) );
          },
declare function={
          fixcdf(\x)= Ncdf(bounded_x(\x,-3,3));
          },
declare function={
          c(\x) = \x*fixcdf( d1(\x) ) - \K * eexp(-\r*\ttm) * fixcdf( d2(\x) );
          },
declare function={
          %c_approx(\x) = \x*\ssigma*sqrt((\ttm))/sqrt((2*pi));
          c_approx(\x) = \x*\ssigma*sqrt((\ttm))/sqrt((2*pi));
          },
spy using outlines={circle, magnification=3, connect spies},
domain=69:91,
scale=0.3,
>=latex
]

  \draw[very thin,color=gray!30] (69,-2)
    grid[step=1] (91,12);

  % The call option price
  \draw[line width=1pt, samples=100] plot (\x, {c(\x)});

  % The approximation
  \draw[densely dotted, domain=75.5:84.5] plot (\x, {c_approx(\x)});

  % The payoff function
  \draw[dashed, samples=100] plot (\x, {upside(\x - \K*exp(-\r*\ttm))});

  %\spy [red, opacity=0.5, size=3cm, samples=200] on (80*0.3, {c(80)*0.3}) in node at (98, 2);
  \spy [red, opacity=0.5, size=3cm, samples=200] on (80*0.3, 2*0.3) in node at (98, 3);

% \def\ssigma{2} \draw[dashed, samples=100] plot (\x, {c(\x)});
%\def\ssigma{1/2} \draw[dashed, samples=50] plot (\x, {c(\x)});
%\def\ssigma{1/3} \draw[dashed, samples=50] plot (\x, {c(\x)});
%\def\ssigma{1/4} \draw[dashed, samples=50] plot (\x, {c(\x)});


\draw[-|] (69,-3) -- (70,-3) node[below] {70};
\draw[-|] (70,-3) -- (75,-3) node[below] {75};
\draw[-|] (75,-3) -- (80,-3) node[below](mid) {80};
\draw[-|] (80,-3) -- (85,-3) node[below] {85};
\draw[-|] (85,-3) -- (90,-3) node[below] {90};
\draw[]   (90,-3) -- (91,-3);
\draw (mid) node[below, anchor=north] {Stock price $S$};

\draw[-|] (68,-2 ) -- ++(0,2)node[left] {0};
\draw[-|] (68, 0 ) -- ++(0,2)node[left] {2};
\draw[-|] (68, 2)  -- ++(0,2)node[left] {4};
\draw[-|] (68, 4)  -- ++(0,2)node[left] {6};
\node (vertmid) at (68, 5) {};
\draw[-|] (68, 6)  -- ++(0,2)node[left] {8};
\draw[-|] (68, 8)  -- ++(0,2)node[left] {10};
\draw    (68, 10)  -- ++(0,2);

\draw (vertmid) node[rotate=90, anchor=south, yshift=12] {Call option price};

\node (eqref1) at (85, 7) {(\ref{eq:calloption:price})};


% Draw a legend
\begin{scope}[xshift=93cm, yshift=12cm]
\footnotesize
\draw[line width=1pt] (0,-1) -- (3.5,-1) node[black,anchor=west] {Call option};
\draw[densely dotted] (0,-2) -- (3.5,-2) node[anchor=west] {Approximation};
\end{scope}
\end{tikzpicture}

\end{document}

%\caption{The approximation to the price of the call option. The dotted line is the approximation, and the solid black line is the call price.}
\end{center}
\end{figure}
%
%
The approximation is only accurate when the difference between the stock price and the strike price is small, and accuracy rapidly deteriorates as this difference grows.
The approximation, however, is so useful that traders know it by heart and are expected to calculate it in their heads (noting that $1/\sqrt{2\pi} \approx 0.4$).
This approximation is also derived in Question 2.20 of \citet{HeardOnTheStreet}, but with more rigour by assuming that the stock price follows arithmetic Brownian motion instead of geometric Brownian motion.
After the derivation, \citet{HeardOnTheStreet} then uses it to answer questions 2.21, 2.22, and 2.23.

In all my interviews, I didn't get many questions about Black and Scholes, stochastic calculus, and derivative pricing.
This is partly because I applied for roles leaning more towards statistics instead of pricing, but it also has to do with the type of jobs advertised at the time of writing.
While there is still work for pricing quants, there has been a stronger focus on modelling roles, helped along by the hype generated by rebranding statistics as machine learning.
There has also been more roles focused on software development, under the label of strategists or \emph{strats}.
Your mileage may vary, I suggest interviewing for some roles, and then to adjust your preparatory material according to what you are tested against.
The classic interview texts,
\cite{HeardOnTheStreet},
\cite{WilmottFAQ}, and
\citet{JoshiQA}
feature plenty of material for derivative pricing roles.



%% Uncomment this if you need references
\bibliography{references,../../src/books}
\bibliographystyle{chicago}


\end{document}
