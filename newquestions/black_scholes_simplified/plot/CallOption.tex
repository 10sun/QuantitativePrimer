\documentclass{standalone}

%https://truonglatex.wordpress.com/2011/11/28/create-standalone-figures-which-are-input-able/
% if you want to input{} this into the maindocument, make sure you do \usepackage{standalone} in there
% That would ignore everything in the preamble of this file
% Makes it easier to maintain, since you can come back and recompile this figure

\usepackage{tikz}


\usetikzlibrary{
  decorations.pathreplacing,
  calc,
  patterns,
  math,
%  datavisualization.formats.functions,
  external,
  spy}

\begin{document}

% >=latex is the specificatin of the arrow tips
% See section 16.4.4 Defining Shorthands
% See pgfmanula 89.3 Syntax for Mathematical Expressions: Functions
% for a set of mathematicsal functions you can use

\def\K{80}
\def\r{0.00}
\def\ttm{30/365}
\def\ssigma{0.2}
\begin{tikzpicture}[
declare function={
          upside(\x)= max(0, \x);
          },
declare function={
          d1(\x)= (ln(\x/\K) + (\r + 0.5*(\ssigma*\ssigma))*(\ttm) )/( \ssigma*sqrt(\ttm));
          },
declare function={
          d2(\x)= d1(\x) - \ssigma * sqrt(\ttm);
          },
declare function={
          % This function guards against the error "margins too large"
          % It ensures x is in [a, b]
          bounded_x(\x,\a,\b) = min(max(\x,\a),\b);
          },
declare function={
          % This function ensures we never send anything with more than 3 decimal
          % places to exp
          eexp(\x)= exp(bounded_x(\x, -6, 6));
          },
declare function={
          % An approximation that will suffice for plotting purposes
          %Ncdf(\x)= 1/(1+eexp(-0.07056*(\x * \x * \x) - 1.5976* (\x) );
          Ncdf(\x)= 1/(1+eexp(-0.07056*(\x*\x*\x) - 1.5976* (\x) );
          },
declare function={
          fixcdf(\x)= Ncdf(bounded_x(\x,-3,3));
          },
declare function={
          c(\x) = \x*fixcdf( d1(\x) ) - \K * eexp(-\r*\ttm) * fixcdf( d2(\x) );
          },
declare function={
          %c_approx(\x) = \x*\ssigma*sqrt((\ttm))/sqrt((2*pi));
          c_approx(\x) = \x*\ssigma*sqrt((\ttm))/sqrt((2*pi));
          },
spy using outlines={circle, magnification=3, connect spies},
domain=69:91,
scale=0.3,
>=latex
]

  \draw[very thin,color=gray!30] (69,-2)
    grid[step=1] (91,12);

  % The call option price
  \draw[line width=1pt, samples=100] plot (\x, {c(\x)});

  % The approximation
  \draw[densely dotted, domain=75.5:84.5] plot (\x, {c_approx(\x)});

  % The payoff function
  \draw[dashed, samples=100] plot (\x, {upside(\x - \K*exp(-\r*\ttm))});

  %\spy [red, opacity=0.5, size=3cm, samples=200] on (80*0.3, {c(80)*0.3}) in node at (98, 2);
  \spy [red, opacity=0.5, size=3cm, samples=200] on (80*0.3, 2*0.3) in node at (98, 3);

% \def\ssigma{2} \draw[dashed, samples=100] plot (\x, {c(\x)});
%\def\ssigma{1/2} \draw[dashed, samples=50] plot (\x, {c(\x)});
%\def\ssigma{1/3} \draw[dashed, samples=50] plot (\x, {c(\x)});
%\def\ssigma{1/4} \draw[dashed, samples=50] plot (\x, {c(\x)});


\draw[-|] (69,-3) -- (70,-3) node[below] {70};
\draw[-|] (70,-3) -- (75,-3) node[below] {75};
\draw[-|] (75,-3) -- (80,-3) node[below](mid) {80};
\draw[-|] (80,-3) -- (85,-3) node[below] {85};
\draw[-|] (85,-3) -- (90,-3) node[below] {90};
\draw[]   (90,-3) -- (91,-3);
\draw (mid) node[below, anchor=north] {Stock price $S$};

\draw[-|] (68,-2 ) -- ++(0,2)node[left] {0};
\draw[-|] (68, 0 ) -- ++(0,2)node[left] {2};
\draw[-|] (68, 2)  -- ++(0,2)node[left] {4};
\draw[-|] (68, 4)  -- ++(0,2)node[left] {6};
\node (vertmid) at (68, 5) {};
\draw[-|] (68, 6)  -- ++(0,2)node[left] {8};
\draw[-|] (68, 8)  -- ++(0,2)node[left] {10};
\draw    (68, 10)  -- ++(0,2);

\draw (vertmid) node[rotate=90, anchor=south, yshift=12] {Call option price};

\node (eqref1) at (85, 7) {(\ref{eq:calloption:price})};


% Draw a legend
\begin{scope}[xshift=93cm, yshift=12cm]
\footnotesize
\draw[line width=1pt] (0,-1) -- (3.5,-1) node[black,anchor=west] {Call option};
\draw[densely dotted] (0,-2) -- (3.5,-2) node[anchor=west] {Approximation};
\end{scope}
\end{tikzpicture}

\end{document}
