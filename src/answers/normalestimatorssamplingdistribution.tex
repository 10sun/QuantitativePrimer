\begin{answer}{normalestimatorssamplingdistribution}
This is a question about sampling theory in the frequentist sense.
If we have an estimator for the parameter, and we draw repeated samples from the population, what will the distribution of the estimator be?
Note that the question isn't Bayesian, since we aren't assuming random parameters; merely that the estimator will have a distribution due to the random sampling of the values it depends on.
The right way to estimate the parameters using the samples are through the sample mean and variance,
\begin{align*}
  \tilde{\mu}_i &= \frac{1}{n}\sum_{j=1}^{n}{y_{ij}} \\
  \tilde{\sigma}_i &=
 \sqrt{ \frac{1}{n-1} \sum_{j=1}^{n}{(y_{ij} - \tilde{\mu}_i)^2} }
 \text{.}
\end{align*}
If we draw histograms of these values, the distribution of the
$\tilde{\mu}_i$
will be normal,
because it is the sum of normal random variables $y_{ij}$ and thus also normal.
The distribution of the
$\tilde{\sigma}_i$
will be chi-squared, because it is the sum of a bunch of squared normal random variables.
The way to prove this is to use the formula for random variable transformation.
If $Z = g(X)$
then
\[
  f_Z(x) =  f_X( g^{-1}(x) )\left| \frac{d}{dx} g^{-1}(x) \right|
\]
where
$f_Z(x)$ is the pdf of $Z$ and
$f_X(x)$ is the pdf of $X$.
When I wrote this down, my interviewer
checked his notes and said
``No, that is wrong, you have to use the following formula'' and produced
\[
  f_Z(z) =  f_X( g^{-1}(z) )\left| \frac{d}{dz} g^{-1}(z) \right|
  \text{.}
\]
Then he asked me to derive it to make sure I understood where it came from.
Here is the derivation.
\begin{align*}
  F_Z(x)
  &=  P(Z < x) \\
  &=  P(g(X) < x) \\
  &=  P(X < g^{-1}(x)) \\
  &=  P(X < g^{-1}(x)) \\
  &=  F_X( g^{-1}(x) ) \\
  f_Z(x) &=  f_X( g^{-1}(x) )\left| \frac{d}{dx} g^{-1}(x) \right|
  \text{.}
\end{align*}
The observant reader will notice that I used $x$ in my derivation, not $z$.
The interviewer nodded and moved on to the next question.
\end{answer}
