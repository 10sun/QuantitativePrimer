\begin{answer}{simplerandomwalkhittingprobability}
This deceptively simple-looking question is difficult and requires specialist theory to solve.
If you don't know the theory required, use the first trick in appendix \ref{ap:tricks} and start with the trivial case.
\index{tricks!start with simplest case}
Here, it means we start the process at $k=1$ and let $l=2$.
Then, the process can go either way with probability $\frac{1}{2}$.
Next, consider $l=3$ where the process can be at $k=1$ or $k=2$.
Let $A_k$ be the event that a process---which is currently at $k$---reaches three before it reaches zero.
The process can only be in two places and they depend on each other in the following manner:
\begin{align*}
P(A_2) &= (0.5)(1) + (0.5)P(A_1)  \\
P(A_1) &= (0.5)(0) + (0.5)P(A_2)
\text{.}
\end{align*}
Write everything in terms of $P(A_1)$ to solve it:
\begin{align*}
P(A_1) &= 0 + (0.5) \Big((0.5)(1) + (0.5) P(A_1)\Big) \\
P(A_1) &= \frac{1}{4} + \frac{1}{4}P(A_1) \\
P(A_1) \left(\frac{3}{4}\right)&= \frac{1}{4} \\
P(A_1) &= \frac{1}{3}
\text{.}
\end{align*}
Then we can also solve for $P(A_2)$:
\begin{align*}
P(A_2) &= \frac{1}{2} + \frac{1}{2} P(A_1)  \\
       &= \frac{1}{2} + \frac{1}{2}\left(\frac{1}{3}\right) \\
       &= \frac{2}{3}
\text{.}
\end{align*}
This is how I started attempting the question and the interviewer seemed happy before I reached a clear answer, then moved on.

For the sake of completeness, let us solve one more case where $l=4$ to see whether a pattern emerges.
Let $A_k$ be the event that a process---which is currently at $k$---reaches four before it reaches zero.
The process can now be in one of three places:
\begin{align*}
P(A_3) &= (0.5)(1) + (0.5)P(A_2)  \\
P(A_2) &= (0.5)P(A_3) + (0.5)P(A_1) \\
P(A_1) &= (0.5)(0) + (0.5)P(A_2)
\text{.}
\end{align*}
Write everything in terms of $P(A_3)$:
\begin{align*}
P(A_2) &= (0.5)P(A_3) + (0.5)\Big(0 + (0.5) P(A_2) \Big)  \\
P(A_2) &= (0.5) P(A_3)  + (0.25) P(A_2)  \\
P(A_2) &= \frac{2}{3} P(A_3)
\end{align*}
\begin{align*}
P(A_3) &= (0.5)(1) + (0.5) P(A_2)   \\
       &= \frac{1}{2} + \frac{1}{2}\left(\frac{2}{3}P(A_3)\right)  \\
            P(A_3) &= \frac{3}{4}
\text{.}
\end{align*}
Use this result to get $P(A_2)$:
\begin{align*}
P(A_2) &= \frac{2}{3} P(A_3)  \\
       &= \frac{2}{3}\left(\frac{3}{4}\right) \\
       &= \frac{1}{2}
\end{align*}
and finally $P(A_1)$:
\begin{align*}
P(A_1) &= \frac{1}{2} P(A_2)        \\
       &= \frac{1}{2}\left( \frac{1}{2} \right) \\
       &= \frac{1}{4}
\text{.}
\end{align*}
It is clear that using this method to find a general solution for $k$ and $l$ would be tedious.
The case where $l=4$, gave the formula
\begin{align*}
P(A_k) &= \frac{k}{4}
\text{.}
\end{align*}

This is question 3.23 from \citet{JoshiQA}, but their question only considers $l=5$.
They solve it using the above method.
It is appealing to suggest that the general answer should be
$P(A_k) = \nicefrac{k}{l}$.
If you write down the equations for the general case, you get
\begin{align*}
P(A_1) &= (0.5)(0)      + (0.5) P(A_2)  \\
P(A_2) &= (0.5) P(A_1)  + (0.5) P(A_3)  \\
P(A_3) &= (0.5) P(A_2)  + (0.5) P(A_4)  \\
 & \vdots \\
P(A_{l-2}) &= (0.5) P(A_{l-3})  + (0.5) P(A_{l-1})  \\
P(A_{l-1}) &= (0.5) P(A_{l-2})  + (0.5)(1)
\end{align*}
which can be rewritten as
\begin{align*}
0 &= (0.5)(0) - P(A_1)   + (0.5)(P(A_2)) \\
0 &= (0.5) P(A_1)  - P(A_2)  + (0.5) P(A_3)  \\
0 &= (0.5) P(A_2)  - P(A_3)  + (0.5) P(A_4)  \\
  & \vdots \\
0 &= (0.5) P(A_{l-3})  - P(A_{l-2}) + (0.5) P(A_{l-1}) \\
0 &= (0.5) P(A_{l-2})  - P(A_{l-1}) + (0.5)(1)
\text{.}
\end{align*}
You are left with something that can also be expressed in matrix form
\begin{align*}
\vec{B}\vec{x} &= \vec{c} \\
  \begin{bmatrix}
 -1           & \frac{1}{2}  & 0            &   \ldots       & 0            & 0            \\
 \frac{1}{2}  & -1           & \frac{1}{2}  &   \ldots       & 0            & 0            \\
 0            & \frac{1}{2}  & -1           &   \ldots       & 0            & 0            \\
  \vdots      & \vdots       & \vdots       &   \ddots       & \vdots       & \vdots       \\
 0            & 0            & 0            &   \ldots       & -1           & \frac{1}{2}  \\
 0            & 0            & 0            &   \ldots       & \frac{1}{2}  & -1           \\
  \end{bmatrix}
  \begin{bmatrix}
  P(A_1) \\
  P(A_2) \\
  P(A_3) \\
  \vdots \\
  P(A_{l-2}) \\
  P(A_{l-1}) \\
  \end{bmatrix}
  &=
  \begin{bmatrix}
    0 \\ 0 \\ 0 \\ \vdots \\ 0 \\ -\frac{1}{2} \\
  \end{bmatrix}
\end{align*}
which can then be used to solve for the vector $\vec{x}$ with the probabilities
\begin{align*}
\vec{x} &= \vec{B}^{-1}\vec{c} \\
  \begin{bmatrix}
  P(A_1) \\
  P(A_2) \\
  P(A_3) \\
  \vdots \\
  P(A_{l-2}) \\
  P(A_{l-1}) \\
  \end{bmatrix}
  &=
  \begin{bmatrix}
 -1           & \frac{1}{2}  & 0            &   \ldots       & 0            & 0            \\
 \frac{1}{2}  & -1           & \frac{1}{2}  &   \ldots       & 0            & 0            \\
 0            & \frac{1}{2}  & -1           &   \ldots       & 0            & 0            \\
  \vdots      & \vdots       & \vdots       &   \ddots       & \vdots       & \vdots       \\
 0            & 0            & 0            &   \ldots       & -1           & \frac{1}{2}  \\
 0            & 0            & 0            &   \ldots       & \frac{1}{2}  & -1           \\
  \end{bmatrix}^{-1}
  \begin{bmatrix}
    0 \\ 0 \\ 0 \\ \vdots \\ 0 \\ -\frac{1}{2} \\
  \end{bmatrix}
\text{.}
\end{align*}
For the case where $l=7$, for example, you have:
\begin{align*}
\vec{B} &=
(-1)
\left[\begin{matrix}1 & - \frac{1}{2} & 0 & 0 & 0 & 0\\- \frac{1}{2} & 1 & - \frac{1}{2} & 0 & 0 & 0\\0 & - \frac{1}{2} & 1 & - \frac{1}{2} & 0 & 0\\0 & 0 & - \frac{1}{2} & 1 & - \frac{1}{2} & 0\\0 & 0 & 0 & - \frac{1}{2} & 1 & - \frac{1}{2}\\0 & 0 & 0 & 0 & - \frac{1}{2} & 1\end{matrix}\right]
\\
\vec{B}^{-1} &=
-
\frac{2}{7}
\left[\begin{matrix}6 & 5 & 4 & 3 & 2 & 1\\5 & 10 & 8 & 6 & 4 & 2\\4 & 8 & 12 & 9 & 6 & 3\\3 & 6 & 9 & 12 & 8 & 4\\2 & 4 & 6 & 8 & 10 & 5\\1 & 2 & 3 & 4 & 5 & 6\end{matrix}\right]
\\
\vec{B}^{-1}\vec{c}
&=
\left[\begin{matrix}\frac{1}{7} & \frac{2}{7} & \frac{3}{7} & \frac{4}{7} & \frac{5}{7} & \frac{6}{7}\end{matrix}\right]^{T}
\text{.}
\end{align*}
The matrix $B$ is nearly diagonal, and I expect there would be a way to prove the expected result using matrix theory, though probably not one that is quick enough to use during an interview.

Martingale theory provides the elegant proof.
The theory is quite heavy and it is unlikely that you will be able to solve questions using martingales without substantial practise.
In my experience interviewers are happy if you attempt the brute force solution; doing this allows you to show how you would break a very difficult problem into simpler parts (and they get to see your algebra).
If you mention anything about martingales on your CV, be ready to use them to solve questions of this type as well as all the questions involving martingales in \citet{JoshiQA} and \citet{HeardOnTheStreet}.

For this question, you must realise that $X_t$ is a martingale, because doing so allows you to do two things: first, to define the times at which the process reaches each of the boundaries as stopping times,
\begin{align*}
  T_0 &= \inf\{ n \geq 0: X_n = 0\} \\
  T_l &= \inf\{ n \geq 0: X_n = l\}
  \text{,}
\end{align*}
and second, because $X_t$ is a martingale, and if you let $T$ be the first time we hit either $0$ or $l$ (therefore meaning $T = \min(T_0 , T_l)$), you get
\[
E(X_T) =  E(X_0) = k
\text{.}
\]
You also have
$X_T=0$ if $T_0 < T_l$
and
$X_T=l$ if $T_0 > T_l$.
The crux is to realise that these two probabilities partition the entire probability space, such that
\begin{align*}
k = E(X_T) &=  0 P(X_T =  0 ) + l P(X_T = l  ) \\
           &=  0 P(T_0 < T_l) + l P(T_0 > T_l) \\
           &=  l P(T_0 > T_l)  \\
P(T_0 > T_l) &= \frac{k}{l}
\text{.}
\end{align*}
The probability $P(T_0 > T_l)$ is the one you are interested in and you get the result that you expected.
\end{answer}
