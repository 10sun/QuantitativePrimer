\begin{answer}{judgescorrectverdict}
The court reaches the correct verdict if either two or three judges are correct.
The obvious solution of $1 - P(\text{All three wrong})$ ignores the scenario in which one judge is correct and the other two are wrong.
You have to write out all the possible combinations:
\begin{align*}
  P(\text{Three correct})  &= p p  (\nicefrac{1}{2}) \\
  P( \text{Two correct} ) &=
     p         p     (1-\nicefrac{1}{2}) +
     p       (1-p)   (\nicefrac{1}{2}) +
   (1-p)       p     (\nicefrac{1}{2}) \\
  &= p^2 - (\nicefrac{1}{2})p^2 + p-p^2 \\
  &= p - (\nicefrac{1}{2})p^2  \\
   P(\text{Correct verdict reached})
  &=
   P(\text{Two or three correct}) \\
  &= (\nicefrac{1}{2})p^2   + p - (\nicefrac{1}{2})p^2  \\
  &=  p
  \text{.}
\end{align*}
Note that you cannot use the binomial coefficients,
$\binom{3}{3}$
and
$\binom{3}{2}$,
as the judges don't all have the same $p$.
\end{answer}
