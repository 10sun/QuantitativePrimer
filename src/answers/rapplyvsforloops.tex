\begin{answer}{rapplyvsforloops}
 To the user, the main difference is in the syntax.
 To the machine, there are many differences in the implementation regarding memory allocation and optimisations.
 The question specifies \verb+lapply+, which applies a function over a list and returns a list.
 Here is some code that will take every item in a list of integers, and apply a function to multiply it by two.
 We also give the for loop for the same action.

\begin{minted}{r}
mult2 <- function(x) return(x*2)

listofintegers <- list(1,2,3)
ans1 <- lapply( listofintegers , mult2)

ans2 <- list() # initialize empty list
for (i in 1:length(listofintegers)){
  ans2[[i]] <- mult2(listofintegers[[i]])
}
\end{minted}
There might be better ways to write the for loop, but this suffices to highlight the difference.
As for the question about preference; you should prefer \verb+lapply+ due to all the optimisation that comes with vectorisation.
The developers of the R created these functions for a reason.
It is also easy to parallelise code written with \verb+lapply+.
The biggest caveat of these functions, however, is readability.
\end{answer}
