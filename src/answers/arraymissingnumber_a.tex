\begin{subanswer}{arraymissingnumber:a}
This is a little harder to figure out.
\index{tricks!start with simplest case}
If you struggle, think about the case with two missing numbers.
Then think of three.
This step-by-step thinking should help you to arrive at the general case.
For many brainteasers, it helps to start with the case of $n=1$ and then generalise, as discussed in appendix \ref{ap:tricks}.
The general solution to this question is similar to the case of one missing number.
You have $n-k$ numbers $x_1, x_2, x_3, \ldots , x_{n-k}$.
To find the $k$ missing values, calculate
\begin{align*}
& \sum_{i=1}^{n-k}{ x_i   } \\
& \sum_{i=1}^{n-k}{ x_i^2 } \\
& \sum_{i=1}^{n-k}{ x_i^3 } \\
& \quad\vdots \\
& \sum_{i=1}^{n-k}{ x_i^k }
\end{align*}
and compare these with their theoretical values.
The result is a system of $k$ equations to solve, with $k$ unknowns.
The answers represent the missing numbers.
For instance, say we have three missing numbers, $m_1, m_2$, and $m_3$, then we can set up the following equations,
\begin{align*}
 \overbrace{\frac{n(n+1)}{2}}^{ \text{Theoretical} \atop \text{value} }
 &=
 \sum_{i=1}^{n-3}{ x_i }
+  m_1 + m_2 + m_3
\\
 \frac{n(n+1)(2n + 1)}{6}
 &=
 \sum_{i=1}^{n-3}{ x_i^2 }
+ m_1^2 + m_2^2 + m_3^2
\\
 \frac{n^2(n+1)^2}{4}
 &=
 \sum_{i=1}^{n-3}{ x_i^3 }
+ m_1^3 + m_2^3 + m_3^3
\text{.}
\end{align*}
We have three equations, and three unknowns.
You will probably have to solve this numerically, possibly by doing an exhaustive search over the integers from $1$ to $n$.
For the theoretical values, you can use Faulhaber's formula:
\[
  \sum_{k=1}^{n}{k^p} = \frac{1}{p+1} \sum_{j=0}^{p}{ (-1)^j \binom{p+1}{j} B_j n^{p+1-j} }
\]
where $B_j$ represents the Bernoulli numbers, or you can calculate them numerically.


\end{subanswer}



