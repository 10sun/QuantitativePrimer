\begin{answer}{derivexpowx}
These questions with the appearance of pure-mathematics usually test two things.
\begin{enumerate}
  \item Rewrite trick: Rewrite the problem to something easier
  \item Solve using integration or differentiation
\end{enumerate}
The first one is the hardest part and if you don't know the specific trick for the specific question, you have to get lucky.
As long as you try \emph{something} to show the interviewer you have the potential to solve the question they should give you some hints to arrive at the rewrite trick.
The second part requires calculus from a year-one mathematics course.
That is a lot of theory to brush up on, but the few rules highlighted in appendix \ref{ap:cribsheet} should be sufficient for most interview questions.

The wrong answer is $(x-1) x^{x-1}$.
Here is the rewrite trick that will put us on the right track,
\begin{align*}
  x^x &= e^{ \ln(x^x )} \\
      &= e^{x \ln(x )}
      \text{.}
\end{align*}
And now we need to use the chain rule and the product rule. Let
\begin{align*}
      e^{x \ln(x )} &=  e^{u} \\
      u &= x \ln(x) \\
     du &= 1 + \ln(x) \quad (\text{Product rule})
\end{align*}
then
\begin{align*}
\frac{d}{dx} e^{x \ln(x )} &= \frac{d}{du} e^{u}  \frac{du}{dx} \\
                           &=  e^{u} (1 + \ln(x)) \\
                           &=  e^{x \ln(x)} (1 + \ln(x)) \\
                           &=  x^x (1 + \ln(x))
\text{.}
\end{align*}
\end{answer}
