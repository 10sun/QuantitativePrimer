\begin{answer}{ncralgorithm}
The question requires us to evaluate
\[
\binom{n}{k} = \frac{n!}{ (n-k)! k! }
\text{,}
\]
thus we need a function to calculate the factorial.
The following code uses recursion to calculate it:
\begin{minted}{python}
def factorial(n):
    if n==0:
        return 1
    return(n * factorial(n-1) )

def nCr(n,k):
    return factorial(n)/(factorial(k) * factorial(n-k))
\end{minted}
For big numbers, this is expensive---a smarter solution exists.
Consider the following example with small integers:
\begin{align*}
\binom{8}{5} &= \frac{8!}{ (8-5)! 5! } \\
             &= \frac{(8 \times 7 \times 6 )}{3!} \\
             &= \frac{8 \times 7 \times 6 }{  3 \times 2 \times 1 }
\text{.}
\end{align*}
Likewise, the general case will be
\begin{align*}
\binom{n}{k}
             &= \frac{n \times (n-1) \times  \ldots \times (k+1) }{ (n-k) \times (n-k-1) \times \ldots \times 1 }
\text{.}
\end{align*}
It is not necessary to call the factorial function three times, since
we can evaluate the nominator and denominator separately:
\begin{minted}{python}
def product(number_list):
    prod = 1
    for number in number_list:
        prod = prod*number

    return prod

def nCr_smart(n,k):
    nominator = product(list(range(k+1,n+1)))
    denominator = product(range(1,n-k+1))

    return nominator/denominator
\end{minted}
By sheer luck, the product function will also work for $0$.
The hardest thing is using the \verb+range+ function of Python correctly, because it starts at the first number, and iterates to one less than the second number.
If the second number is smaller than the first, it returns an empty list \verb+[]+.
You can further optimise by noting that $\binom{n}{k}=\binom{n}{n-k}$ and always choosing to evaluate the one that is less work (for the function above, large values of $k$ are preferred).
\end{answer}
