\begin{answer}{nameafewnonlinearmodels}
Some examples of non-linear models are:
\begin{itemize}
  \item Survival models, which usually study a non-linear hazard rate $h(t)$
  \item Generalised Linear Models (GLM), which can have non-linear link functions
  (the logistic regression uses the logit link function, and the Gamma GLM uses the log link function)
  \item Gaussian processes (and other non-parametric regressions)
  \item Neural networks, which are an approximation of Gaussian processes, as shown by \citet[chap.~15.4]{murphy2012machine} using the proof that was first pointed out by  \citet{neal1996bayesian}: as the number of hidden layers in a neural net goes to infinity, it becomes a Gaussian process
\end{itemize}
In fact, a broad class of non-linear models can be presented in the following way:
\begin{align*}
  y \sim
  \operatorname{MultivariateNormal}( g(X,\theta), \Sigma )
\end{align*}
where $g(X,\theta)$ is some non-linear function of the parameters and the inputs.
You can make this more general by replacing the multivariate normal distribution,
\begin{align*}
  y \sim Dist( g(X,\theta) )
\end{align*}
where $y$ is assumed to follow some distribution and $g(X, \theta)$ is some function that links the inputs to the parameters of the distribution.
See \citet{murphy2012machine} for an exhaustive treatment of this idea.

\end{answer}
