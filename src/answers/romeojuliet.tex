\begin{answer}{romeojuliet}
This is question 3.24 from \citet{JoshiQA}.
It is easy to overcomplicate things by thinking in great detail about what happens at the end of the interval.
For instance, what if Romeo arrives at 08:58? He will wait 15 minutes, but Juliet cannot arrive after 09:00.
The crux, however, is to realise that they will only meet \emph{if they arrive within 15 minutes of each other}.
Let the interval $[0,1]$ represent the hour between 08:00 and 09:00, and let $x$ be the time Romeo arrives, and $y$ be the time Juliet arrives.
They will only meet if
\[
  |x - y| < \frac{15}{60} = \frac{1}{4}
  \text{.}
\]
We have
$x$ and $y$
independently and identically distributed
$\text{Uniform}(0,1)$.
Thus, we need to find $P(\abs{x - y} < \nicefrac{1}{4})$.
It is almost certain that the interviewer does \emph{not} want you to solve this using integration,
but rather visually on the unit square.
\index{tricks!unit square integration}
This is a standard interview trick (described in appendix \ref{ap:tricks}).
When $x<y$, you are in the top-left half of the square, and you need to consider
\begin{align*}
  y-x &< \nicefrac{1}{4} \\
  y &< x + \nicefrac{1}{4}
  \text{.}
\end{align*}
When $x \geq y$, we are in the bottom-right half of the square, and we need to consider
\begin{align*}
  x-y &< \nicefrac{1}{4} \\
  y  &> x -\nicefrac{1}{4}
  \text{.}
\end{align*}
The answer is the shaded portion of the square, which can be calculated quickly by breaking the square down into pieces (see below), revealing the answer as $\nicefrac{7}{16}$.
\begin{center}

\begin{tikzpicture}[scale=6, domain=0:1]
\draw (0,0)     --  (0,1) node[midway, left] {$y$}-- (1,1)  -- (1,0) -- (0,0) node[midway, below] {$x$};
\filldraw[pattern=dots, pattern color=lightgray]
(0    , 0.25) --
(0.75 ,   1 ) node[above,midway, sloped] {$y=x + \nicefrac{1}{4}$} --
(1    ,  1     ) --
(1    ,  0.75  ) --
(0.25 ,  0  ) node[below,midway, sloped] {$y=x - \nicefrac{1}{4}$}
              node[below] {$\frac{1}{4}$} --
(0 ,  0  ) --
(0    , 0.25) node[left] {$\frac{1}{4}$};
\draw[dashed] (0,0) -- (1,1);

\begin{scope}[xshift=1.5cm, yshift=0.5cm, every node, scale=0.5]
\filldraw[pattern=dots, pattern color=lightgray]
(0    , 0.25) --
(0.75 ,   1 ) --
(1    ,  1  ) --
(1    ,  0.75  ) --
(0.25 ,  0  ) --
(0 ,  0     ) --
(0    , 0.25) ;
\draw[xshift=-1pt, yshift=1pt]
(0 , 0.25) -- (0,1) -- (0.75, 1) -- (0 , 0.25) ;

\draw[xshift=1pt, yshift=-1pt]
( 0.25, 0) -- (1,0) -- (1, 0.75) -- (0.25, 0) ;

\draw[xshift=2pt, yshift=-1pt, decorate, decoration={brace, mirror, amplitude=13pt}]
(1,0) -- (1,0.75) node [midway, xshift=+17pt]{$\frac{3}{4}$};
\end{scope}

\begin{scope}[xshift=1.2cm, yshift=0.1cm, every node, scale=0.3]
\filldraw[pattern=dots, pattern color=lightgray]
(0    , 0.25) --
(0.75 ,   1 ) --
(1    ,  1  ) --
(1    ,  0.75  ) --
(0.25 ,  0  ) --
(0 ,  0     ) --
(0    , 0.25) ;

\node at (1.2, 0.4) {$=$};
\node at (1.7, 0.4) {$1$  $-$};
\node at (1.7, -0.8) {
$ \begin{aligned}
 &= 1 -\left( \frac{3}{4}\right)^2 \\
 &= 1 - \frac{9}{16} \\
 &= \frac{7}{16}
\end{aligned}$
};
\end{scope}

\begin{scope}[xshift=1.8cm, yshift=0.1cm, every node, scale=0.3]
\draw[xshift=3pt, yshift=-3pt]
(0 , 0.25) -- (0,1) -- (0.75, 1) -- (0 , 0.25) ;
\draw[xshift=-3pt, yshift=3pt]
( 0.25, 0) -- (1,0) -- (1, 0.75) -- (0.25, 0) ;
\end{scope}
\end{tikzpicture}

\end{center}
You can use similar logic to determine what the probability would be if they each wait $t$ minutes or if Romeo waits twice as long as Juliet.
\end{answer}
