\begin{subanswer}{sqlhighestsalary}
There are many ways to do this, involving merges, sorts and joins.
I find the easiest and clearest way is to use \verb+LIMIT+ and \verb+OFFSET+, which exist in all SQL implementations.
The logic is to sort descending by salary, then offset the rows by one, and return only one entry.
The result will be the second highest entry.
You can easily change the offset to return the $n$th highest salary:
\begin{minted}{sql}
SELECT
    Employee_name,
    Department,
    Salary
FROM salaries
ORDER BY Salary DESC
LIMIT 1
OFFSET 1
\end{minted}
Appendix \ref{ap:sqlite} contains code to load the table and test this query.
\end{subanswer}
