\begin{answer}{drop20cards}
The fact that 20 cards were randomly discarded before you got to draw two has no influence on your probability of drawing aces.
This is not obvious from the way the question is framed.
Getting this question over the phone may be a hint that the answer can't be too complicated, but this is not a rule you can rely on.


One way to think about it is to pretend we are randomly dealing the deck into two groups: one group has two cards, and one group has 20.
The fact that I deal the 20 cards first doesn't affect the probability of the second group.
Another way to think about it is from the perspective of a poker game, like Texas Hold'em, where each player gets 2 cards.
You can calculate the probability of getting two aces before the cards are dealt.
Whether you are playing against two or ten other players, it doesn't affect \emph{your} probability of getting two aces on the deal.
Consider the 20 cards that are randomly being discarded in this question as the ones given to your opponents in a game.
You wouldn't think it was unfair if your opponents got their cards \emph{before} you receive yours, because it is still dealt random and thus your intuition doesn't trigger alarms.
The way the question is worded expresses this scenario as something less intuitive and you'll need to reason it out before answering.


Questions about card drawing should be answered using the combination formula.
It is a useful technique since you can answer very complicated card-related questions with it.
For instance, if I draw 6 cards, what is the probability of drawing exactly two aces and one king?
The answer is
\begin{align}
 \frac{\binom{4}{2} \binom{4}{1} \binom{44}{3} }{\binom{52}{6} }
\end{align}
where
\begin{align*}
\binom{4}{2}&
  \text{ is the number of ways to draw 2 aces from the 4 in the deck,} \\
\binom{4}{1}&
  \text{ is the number of ways to draw 2 kings from the 4 in the deck,} \\
\binom{44}{3}&
  \text{ is the number of ways to draw  any 3 cards from remaining 44 in the deck,} \\
\binom{52}{6}&
  \text{ is the number of ways to draw any 6 cards from a deck of 52.}
\end{align*}
Review the probabilities of card drawing as part of interview preparation; they aren't difficult but require some practise.
For the current question you only draw two cards, and seek the probability that they both are aces:
\begin{align}
 \label{eq:drop20cards:cardmath}
P(\text{Drawing two aces})
&= \frac{\binom{4}{2} }{\binom{52}{2} } \\
&= \frac{ \frac{4!}{(2!)(2!)}  }{ \frac{52!}{ (50!) (2!) } } \nonumber \\
&= \frac{ 4 \times 3  }{ 52 \times 51 } \nonumber
\text{.}
\end{align}
For this question, there is another method:
\begin{align*}
P(\text{Drawing two aces})
&=
P(\text{First card is an ace})
\times
P(\text{Second card is an ace})
\\
&=
\frac{4}{52}
\times
\frac{3}{51}
\end{align*}
and then you need to simplify the fraction in your head, \emph{quickly}:
\begin{align*}
\frac{4}{52}
\times
\frac{3}{51}
=
\frac{1}{13}
\times
\frac{1}{17}
=
\frac{1}{221}
\text{.}
\end{align*}
During my interview, I first tried to use the method in
\eqref{eq:drop20cards:cardmath},
but from the incredulity and confusion in my interviewer's voice he probably was not familiar with the technique.
Many interviewers who ask statistics and probability only took one brief statistics course and---outside of some brainteasers they looked up---aren't aware of all the techniques and nuances, nor the differences in jargon.\footnote{This has been exacerbated by the redefinition of many statistical concepts within Machine Learning.}
Fortunately I knew about both methods, but under pressure I was unable to perform the mental arithmetic fast enough.
\end{answer}
