\begin{answer}{ar1processmeanandvar}
There is a long answer and a short answer.
If you know the short answer, it is probably best to give it first.
If your interviewer is unsatisfied, you can give the long solution.
Start by assuming
\begin{equation}
\label{eq:e_and_pi:constantassumption}
\begin{aligned}
\E(Y_i) &= \E(Y_j) \\
    &\text{and}    \\
\Var(Y_i) &= \Var(Y_j)
\end{aligned}
\end{equation}
for all $i = j$.
A more formal setting would require a more rigorous proof, but the interviewer might be happy to let you use this assumption for now, given the fact that their question implies it.
You can also point out that, for an AR(1) process, you know it is wide-sense stationary when
$|\alpha_1| < 1$,
but the interviewer might still want proof.
Using the assumption in (\ref{eq:e_and_pi:constantassumption}), you can write
$\E(Y_i) = \E(Y_j) = \mu$, then solve for $\mu$
\begin{align*}
  \E(Y_t) &= \E(\alpha_0 + \alpha_1 Y_{t-1} + \varepsilon_{t}) \\
  \E(Y_t) &= \alpha_0 + \alpha_1 \E(Y_{t-1}) + \E(\varepsilon_{t}) \\
   \mu    &= \alpha_0 + \alpha_1   \mu \\
   \mu    &= \frac{ \alpha_0}{1-\alpha_1}
   \text{.}
\end{align*}
Similarly, for the variance,
$\Var(Y_i) = \Var(Y_j) = \sigma^2$ and you can solve for $\sigma^2$
\begin{align*}
  \Var(Y_t) &= \Var(\alpha_0 + \alpha_1 Y_{t-1} + \varepsilon_{t}) \\
  \Var(Y_t) &=  \alpha_1^2 \Var(Y_{t-1}) + \Var(\varepsilon_{t}) \\
  \sigma^2  &=  \alpha_1^2 \sigma^2 + \sigma^2_{\varepsilon_{t}} \\
  \sigma^2  &=  \frac{ \sigma^2_{\varepsilon_{t}}}{1- \alpha_1^2}
  \text{.}
\end{align*}

During my interview I was not aware of the short way, but the interviewer seemed content with the answer below---which is the long answer.
We have to expand the process,
\begin{align*}
  Y_t &=
  \alpha_0 + \alpha_1
  \textcolor{blue}{Y_{t-1}} + \varepsilon_{t} \\
      &=
  \alpha_0 + \alpha_1
  \textcolor{blue}{(
  \alpha_0 + \alpha_1 Y_{t-2} + \varepsilon_{t-1}
  )}
                       + \varepsilon_{t} \\
      &=
  \alpha_0 + \alpha_1
  \alpha_0 + \alpha_1^2 \textcolor{dwred}{Y_{t-2}} + \alpha_1\varepsilon_{t-1}
                       + \varepsilon_{t} \\
      &=
  \alpha_0 + \alpha_1
  \alpha_0 + \alpha_1^2
  \textcolor{dwred}{(
  \alpha_0 + \alpha_1 Y_{t-3} + \varepsilon_{t-2}
  )
  }
                         + \alpha_1\varepsilon_{t-1}
                        + \varepsilon_{t} \\
      &=
  \alpha_0 + \alpha_1
  \alpha_0 + \alpha_1^2
  \alpha_0 + \alpha_1^3 \textcolor{dwgreen}{Y_{t-3}} + \alpha_1^2\varepsilon_{t-2}
                        + \alpha_1\varepsilon_{t-1}
                        + \varepsilon_{t} \\
      &=
  \alpha_0 + \alpha_1
  \alpha_0 + \alpha_1^2
  \alpha_0 + \alpha_1^3
  \textcolor{dwgreen}{(\ldots)} + \alpha_1^2\varepsilon_{t-2}
                        + \alpha_1\varepsilon_{t-1}
                        + \varepsilon_{t} \\
      &=
  \alpha_0
  \sum_{k=0}^{\infty}
  {\alpha_1^k}
  +
  \sum_{k=0}^{\infty}{
    \alpha_1^k
    \varepsilon_{t-k}
    }
    \text{.}
\end{align*}
We only had to work to
$Y_{t-3}$
to notice the pattern.
Taking the expectation yields
\begin{align*}
\E(Y_t)
&=
\E\left(
  \alpha_0
  \sum_{k=0}^{\infty}
  {\alpha_1^k}
  +
  \sum_{k=0}^{\infty}{
    \alpha_1^k
    \varepsilon_{t-k}
    }
    \right) \\
&=
  \alpha_0
  \sum_{k=0}^{\infty}
  {\alpha_1^k}
  +
  \sum_{k=0}^{\infty}{
    \alpha_1^k
\E\left(
    \varepsilon_{t-k}
    \right)
    } \\
&=
  \alpha_0
  \sum_{k=0}^{\infty}
  {\alpha_1^k}
  \text{.}
\end{align*}
This is a geometric series, which converges to
\begin{align*}
\E(Y_t)
&=
  \frac{\alpha_0}{1-\alpha_1}
\end{align*}
if
$|\alpha_1| < 1$.
Likewise, for the variance we have
\begin{align*}
\Var(Y_t)
&=
\Var\left(
  \alpha_0
  \sum_{k=0}^{\infty}
  {\alpha_1^k}
  +
  \sum_{k=0}^{\infty}{
    \alpha_1^k
    \varepsilon_{t-k}
    }
    \right) \\
&=
  \sum_{k=0}^{\infty}{
    (\alpha_1^k)^2
\Var\left(
    \varepsilon_{t-k}
    \right)
    } \\
&=
  \sum_{k=0}^{\infty}{
    (\alpha_1^2)^k
    \sigma^2_{\varepsilon}
    } \\
&= \frac{ \sigma^2_{\varepsilon_{t}}}{1- \alpha_1^2}
\end{align*}
since the series converges if
$|\alpha_1^2| < 1$,
which will be the case if
$|\alpha_1| < 1$.

This question allows the interviewer to test your algebra, and also your familiarity with the AR(1) process, which is the simplest of the ARIMA class of models.
If you mentioned time series on your CV---or if the role requires knowledge about time series---make
sure you can write down the equations for a AR(p),  MA(q), ARMA(p,q), and GARCH(p,q) processes.
They are included in appendix \ref{ap:cribsheet}.
\end{answer}
